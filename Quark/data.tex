%!TEX root = data.tex
\documentclass[]{article}
\usepackage{graphicx}
\usepackage{tikz}
\usepackage{xspace}
\usepackage{enumitem}
\usepackage{mathtools}
\usepackage{cleveref}
\usepackage{listings}
\usepackage{amsmath}
\usepackage{amssymb}
\usepackage{amsthm}
\usepackage[utf8]{inputenc}
\usetikzlibrary{calc,automata,positioning}
\def\istari{\textsc{Istari}\xspace}
\usepackage{float}
%opening
\input{defs}
\title{Data for Verification of Quantitative Network Properties}

\author{Kausik Subramanian}

\begin{document}

\maketitle

\section{Network and Configuration Data}
\begin{itemize}
\item The Microsoft datacenter configurations used in ARC~\cite{arc}.
\item Bandwidth capacity of the network links.
\end{itemize}

\section{Traffic Data Format}
\begin{itemize}
	\item Traffic matrix $T$ of all Source-Destination IP pairs where $T_{ij}$ = traffic rate
	from source $i$ to destination $j$. For each pair, we would need the ingress router and egress
	router (though we could reconstruct this from the configurations). 
	\item Multiple traffic matrices (due to traffic variations) could also be used in the analysis. 
	If actual packet traces exist with the following data, we could construct the traffic matrices
	from the data: 
	\begin{center}
	Timestamp $|$ Source IP $|$ Destination IP $|$ Ingress Edge Router $|$ Egress Edge Router $|$ Packet Size
	\end{center}
\end{itemize}

\section{Failure Data}
\begin{itemize}
	\item Individual Link Failure Data to compute link failure probabilities: 
	\begin{center}
	Link $|$ Time link was down in a time period (day/week/month)
	\end{center}
	\item Link failure correlations: Links whose failure depends on other links' failures, 
	if such correlations exist.
\end{itemize}

	


\bibliographystyle{abbrv} 
\begin{small}
	\bibliography{refs}
\end{small}

\end{document}