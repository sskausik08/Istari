%!TEX root = paper.tex
\documentclass[]{article}
\usepackage{graphicx}
\usepackage{tikz}
\usepackage{xspace}
\usepackage{enumitem}
\usepackage{mathtools}
\usepackage{cleveref}
\usepackage{listings}
\usepackage{amsmath}
\usepackage{amssymb}
\usepackage{amsthm}
\usepackage[utf8]{inputenc}
\usetikzlibrary{calc,automata,positioning}
\def\istari{\textsc{Istari}\xspace}
\usepackage{float}
%opening
\input{defs}
\title{Distributed Verification of Quantitative Network Properties}

\author{Kausik Subramanian, Loris D'Antoni, Aditya Akella}

\begin{document}

\maketitle

\begin{abstract}
We investigate distributed verification and synthesis in 
legacy networks related to quantitative network policies.
\end{abstract}

\section{Introduction}
One of the key challenges of designing networks is verifying
how good the network performs. This is further complicated 
by the fact that these networks are highly prone to failures
and variable traffic demands. One of the common failure scenarios
used to evaluate the network design and strategy is to quantify 
the relevant metrics like congestion under the occurence  
of $t$ arbitrary link failures. While this is effective to understand
the worst-case performance of the network, the probability
of two links failing need not be uniform, and thus, we would 
want to estimate the expected value of the metric, given
link failure probabilities. Similarily, the traffic characteristics
modelled as a distribution can help us provide a realistic
metric to evaluate the network, rather than using the tail
characteristic for worst-case analysis.

\section{Limitations of Existing Work}
Two works try to answer some of the questions pertaining to
quantitative properties of the network. In this section, we 
will describe the approach and their limitations.

\paragraph{Probabilistic NetKAT}
Probabilistic NetKAT~\cite{probnetkat, probnetkatpopl17} 
is an extension of NetKAT with support
for probabilistic routing behaviors, for e.g., a switch 
can forward traffic to one of its neighbors based on a 
probability distribution. This can be used to express different
routing strategies like ECMP, multi-path routing and 
oblivious TE. The language supports queries 
on these programs like expected congestion and latency. ProbNetKAT
handles link failures by modelling the link as a program
which probabilistically drops a packet (link failure
probability).

However, the ProbNetKAT language is not powerful enough to 
express more complex routing strategies which react to failures. 
The simplest example is a OSPF network, where traffic is 
forwarded along the shortest weighted path between the endpoints. 
When a link fails, the network recomputes the paths based on the 
functioning network topology. ProbNetKAT cannot be used to express
the OSPF routing strategy. Traffic engineering strategies which adapt 
to failures to minimize an objective like MPLS tunneling also 
cannot be expressed using ProbNetKAT. Also, it is not clear 
if the approaches used in ProbNetKAT scale to larger networks 
and workloads.

\paragraph{Robust Validation of Network Designs}
Chang et. al~\cite{robustvalidation} propose a framework 
to validate network design under failures and uncertain demands. 
The framework formulates the validation problem as an 
optimization problem: to find the maximum value a metric can
achieve under a set of link failure scenarios. An example 
metric is the Maximum Link Utilisation. Importantly,
this framework allows the specification of routing strategies
which adapt based on the current network topology: Multi-commodity
flow routing and MPLS-style tunneling. 

The limitation of this framework is that it performs a 
worst-case analysis of the network by finding the maximum value 
achieved by the metric. For example, if worst-case 
MLU of the network is $> 1$,
it means that under a certain set of link failures, the network is
oversubscribed. However, the occurrence of this particular scenario
could be highly improbable, and the cost of repairing the 
over subscription may not be reap significant benefits. 
Thus, an useful metric
to validate the network design is to 
compute expected MLU based on the 
link failure probability distributions. 
Computing expected 
MLU in the framework 
would incur an exponential blowup due to the
enumeration of failure scenarios.

\section{Problem Statement (Old)}
\paragraph{Input.} We are provided as input a 
network topology $T=(V,L)$, an adaptive 
routing strategy $R(T_f)$
parameterized over the current functioning topology $T_f$
which decides the paths taken by the traffic classes
in $T_f$. Let us define the 
set of traffic classes $\Lambda 
\subseteq V \times V$. For each $\lambda \in \Lambda$,
we are given a traffic probability 
distribution function 
$\omega(\lambda)$
(discrete or continuous?). To quantify behavior under
failures, we have a link failure probability function
$\mu: L \rightarrow [0,1]$. Let us define $X$ to be 
the set of failure scenarios (bounded by $k$ links).

\paragraph{Output.} Let us define the
metric $M$ which takes an input a topology and a routing 
strategy; it returns a 
real number (for example, max link utilization). 
The question is as follows: for a given $T$ and $R$,
can we compute: 
\[
	E[M] = \sum_{x \in X} \mu(x). M(T_x, R(T_x))
\]

\paragraph{Verification Approach.} The naive brute-force 
approach is to enumerate all link failures $x \in X$
and compute $M(T_x, R(T_x)$, which can then be aggregated to 
compute $E[M]$. This process is highly parallelizable (each 
$x$ can be potentially run on a different machine). Thus, we
propose a distributed verification framework to speed and 
scale up the queries.  

Given $c$ compute machines, we need to partition
the set of failures $X$ across
 the machines. However, the partitions themselves
 can be large; we cannot enumerate
 every possible failure scenario in the set. 
 Thus, we need to partition the set of all failures
 to form clusters based on the $M$ value 
 and approximate $E[M]$ with 
 provable guarantees on the distance from the actual 
 value computed for the partition. The other aspect 
 is to leverage the link failure probability function
 $\mu$ to avoid enumerating all possible link failures.
 
  

\paragraph{Synthesis.} While it is useful for operators
to compute metrics to validate network design, operators need
to repair the network (by changing routing strategies, or 
provisioning new links) to ensure the metric does not violate
a certain threshold (for e.g. a cloud provider can provide 
an SLO of 99.99\% availability). Thus, we investigate 
if we can suggest repairs along with the verification computation
in a distributed manner. 
 
\section{Identifying Most Likely Most Impacting Failure Scenarios}
\paragraph{Input.} Routing configurations which can be parsed
using Batfish/Minesweeper. Link failure probabilities (Note: 
two link failure probabilities could be dependent) is obtained
from real-world data. 

\paragraph{Policies.} Use Minesweeper/Batfish/ARC qualitative policies.
For quantitative policies, we can use traffic matrices distributions 
(refer to SOL) for checking congestion-freedom under failures. 

\paragraph{Output.} Given a set of policies $P$, let $W(P, T)$ 
be the penalty incurred by the topology $T$ with the failed links 
(this metric indicates how many policies are violated in the 
current network topology). We want to find the top-$l$ failure scenarios
with probability greater than threshold $\epsilon$ with respect to 
the penalty incurred under each topology with the failed links ($< k$ link
failures).

\bibliographystyle{abbrv} 
\begin{small}
	\bibliography{refs}
\end{small}

\section{Problem Formulation}
\noindent\paragraph{Input}
\begin{itemize}
\item Network Topology $N = (V, L)$: set of routers and links. 
\item ETGs constructed from network configurations $C$ (OSPF/BGP/Static/ACLs/...)
\item Set of flows where 
each flow $f \in FLOWS$ is of the form: $<$flow, srcRouter, dstRouter, traffic$>$\
\item Link Probability Failures $P_f(r_1, r_2)$
\end{itemize}
\noindent\paragraph{Variables}
\begin{itemize}
\item Link Failure Variable $F(r_1, r_2)$ which is set to 1 if link $r_1 - r_2$ fails.

\item Best Route $R_{best}(r, f)$: This variable is used to store the shortest route  
of all input routes received at router $r$ for flow $f$.

\item Network Forwarding $Fwd$: The control plane will decide the next-hop 
router based on the best route received at a router.  
$Fwd(r_1, r_2, f)== 1$ implies that router $r_1$ will forward flow $f$ to router $r_2$.

\item Link Congestion $\omega(r_1, r_2)$: Binary variable indicating if a 
link is congested or not.  
%We can model Congestion of link $r_1 \rightarrow r_2$ as $C(r_1, r_2) = \frac{}{\sum Fwd(r_1, r_2, p) * Traffic(p)$ 
\end{itemize}

\noindent\paragraph{Constraints}
We present the integer linear constraints to verify if the network 
is congestion-free under $k$ arbitary link failures. This is 
enforced by the following constraint:
\begin{equation}
	\forall r_1, r_2.~(r_1,r_2) \in L. ~\sum_{(r_1, r_2)} F(r_1, r_2) \leq k
\end{equation}

For a flow $f$, at any vertex $v$ of the ETG, the best route to the
destination will the shortest active path through of its neighbors ($N(v,f)$).
We can inductively express this as: 
\begin{equation}
	\forall n \in N(v,f).~ R_{best}(v, f) \leq R_{best} (n, f) + W(v, n, f) + \infty \times F(v,n) 
\end{equation}
where $W(v,n,f)$ denotes the ETG edge weight and $\infty$ is a large constant. These
constraints ensure $R_{best}(v, f)$ is the smallest of all the active routes at $v$.

The constructed ETG have the following property: forwarding happens along the
active shortest path. Thus, forwarding will happen along the link which sends
the best advertisement (shortest in this case). This is expressed as follows:
\begin{equation}
	\forall n \in N(v,f). ~R_{best}(v,f) + \infty \times (1 - Fwd(v,n,f)) \geq R_{best} (n, f) + W(v, n, f)
\end{equation}
These constraints ensure that $Fwd(v,n,f)$ is 1 along the best path.

Since forwarding cannot happen along failed links, we add the following 
constraints to ensure correct forwarding semantics:
\begin{equation}
\forall n \in N(v,f).~Fwd(v,n,f) + F(v,n) \leq 1 
\end{equation}
Thus, if link $v-n$ is failed, flow $f$ will not be forwarded through the link.

\noindent\paragraph{Output} 
To check for congestion, we first add the following constraints for 
each link to indicate congestion when utilization exceeds capacity: 
\begin{equation}
\forall (r_1, r_2) \in L. ~~\frac{\sum_{f \in F} Fwd(r_1, r_2, f) * TC(f)}{Cap(r_1, r_2)} - \omega(r_1, r_2) \geq 0
\end{equation}
Thus, if traffic on link does not exceed capacity of the link, the congestion value
omega will be 0. Also, if $\omega$ is 1, then the link has to be congested. Therefore,
to check for a particular link congestion, we can have the objective as:
\[
\texttt{maximize} ~~\omega(r_1, r_2)
\]	
To find if any link in the network can become congested, we can add the following
objective:
\[
\texttt{maximize} ~~\sum_{\forall (r_1,r_2) \in L} \omega(r_1, r_2)
\]


\textbf{Most Likely Critical Event}: We modify our question of congestion from above to
incorporate failure probabilities for link failure scenarios $(r_1, r_1'),\ldots(r_{k-1}, r_{k-1}')$ : 
\[
\texttt{maximize} (P_f(r_1,r_1') *\ldots P_f(r_{k-1}, r_{k-1}')). 
~~\exists r_1, r_2. \omega(r_1, r_2) = 1
\]
Product can be expressed as a sum of logarithms of probabilities to ensure the objective is linear. 

%%!TEX root = paper.tex
\begin{theorem}
Given a topology $T = (R, L)$, the ARC, the set of  
traffic characteristics $\{tc \in TC | tc \in IP \times R \times R \times \nat\}$ and 
$(L_f, \fail)$ where $L_f \subseteq L$ denotes the subset of links 
which can potentially fail and $\fail$ denotes the number of links which have failed,
the problem of finding a set of links $\links \subseteq L_f$ such that $|\links| = \fail$
and there exists a link $l \in L \setminus \links$ where the sum of flows exceeds capacity of the link $l$
\end{theorem}

\begin{proof}
We show that the Hamiltonian path problem, which is NP-complete reduces to the
quantitative verification problem under failures. The latter is also in NP, so after the reduction we 
 can conclude that it is also NP-complete.
 
\end{proof} 


\end{document}