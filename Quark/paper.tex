%!TEX root = paper.tex
\documentclass[]{article}
\usepackage{graphicx}
\usepackage{tikz}
\usepackage{xspace}
\usepackage{enumitem}
\usepackage{mathtools}
\usepackage{cleveref}
\usepackage{listings}
\usepackage{amsmath}
\usepackage{amssymb}
\usepackage{amsthm}
\usepackage[utf8]{inputenc}
\usetikzlibrary{calc,automata,positioning}
\def\istari{\textsc{Istari}\xspace}
\usepackage{float}
%opening
\newcommand{\minisection}[1]{\noindent{\bf #1.}}
\newcommand{\secref}[1]{{\S\ref{#1}}}
\newcommand{\secsref}[2]{{Sections~\ref{#1} and \S\ref{#2}}}
\newcommand{\figref}[1]{{Figure~\ref{#1}}}
\newcommand{\figsref}[2]{{Figure~\ref{#1} and \ref{#2}}}
\newcommand{\appref}[1]{{Appendix~\ref{#1}}}
\newcommand{\thmref}[1]{{Theorem~\ref{#1}}}
\newcommand{\lemref}[1]{{Lemma~\ref{#1}}}
\newcommand{\corref}[1]{{Corollary~\ref{#1}}}
\newcommand{\proref}[1]{{Property~\ref{#1}}}
\newcommand{\tabref}[1]{{Table~\ref{#1}}}

\newcommand{\compactcaption}[1]{\vspace{-0.5em}\caption{#1}\vspace{-0.5em}}
\newcommand{\botcompactcaption}[1]{\caption{#1}\vspace{-1em}}
\newcommand{\topcompactcaption}[1]{\vspace{-1em}\caption{#1}\vspace{-0.5em}}

%\newtheorem{definition}{Definition}

\newenvironment{compactitemize}
{
   \begin{itemize}[leftmargin=1.5em]
   \vspace{-1ex}
   \setlength{\topsep}{0pt}
   \setlength{\itemsep}{0em}
   \setlength{\parskip}{0pt}
   \setlength{\parsep}{0pt}
}
{
   \vspace{-1ex}
   \end{itemize}
}
\newenvironment{compact2itemize}
{
	\begin{itemize}[leftmargin=1.5em]
		\vspace{-1ex}
		\setlength{\topsep}{0pt}
		\setlength{\itemsep}{0.5em}
		\setlength{\parskip}{0pt}
		\setlength{\parsep}{0pt}
	}
	{
		\vspace{-1ex}
	\end{itemize}
}


\newenvironment{compactenumerate}
{
   \begin{enumerate}[leftmargin=1.5em]
   \vspace{-1ex}
   \setlength{\topsep}{0pt}
   \setlength{\itemsep}{0em}
   \setlength{\parskip}{0pt}
   \setlength{\parsep}{0pt}
}
{
   \vspace{-1ex}
   \end{enumerate}
}

\def\rat{\mathbb{Q}}
\def\real{\mathbb{R}}
\def\nat{\mathbb{N}}
\def\route{\mathfrak{R}}
\def\paths{\mathfrak{P}}
\def\links{\mathbb{L}}
\def\waypt{\mathbb{W}}


%-----
%Space tricks

%\algtext*{EndWhile}% Remove "end while" text
%\algtext*{EndIf}% Remove "end if" text
%\algtext*{EndProcedure}% Remove "end while" text
%\algtext*{EndFor}% Remove "end while" text

\makeatletter
\lst@InstallKeywords k{attributes}{attributestyle}\slshape{attributestyle}{}ld
\makeatother
\newcommand\gtt[1]{{\color{gray}\texttt{\textbf{#1}}}}
\newcommand\nf[1]{{\normalfont#1}}

\lstdefinestyle{customc}{
  belowcaptionskip=1\baselineskip,
  breaklines=true,
  xleftmargin=\parindent,
  language=C,
  showstringspaces=false,
  basicstyle=\small\ttfamily,
  keywordstyle=\small\bfseries\ttfamily\color{NavyBlue},
  commentstyle=\itshape\color{black},
  identifierstyle=\ttfamily\color{black},
  stringstyle=\itshape\color{NavyBlue},
  keywords={ map, flatMap, reduce, return, define, gaussian, gauss, step, elif, then, skip, if, else, reduceByKey, sensitiveAttribute, fairnessTarget, def},
moreattributes={ let, where}, % etc...
attributestyle = \itshape\ttfamily\color{attributecolor}
}

\setlist[itemize]{
    %labelindent=.4cm,
    %style=unboxed,
    %leftmargin=.4cm,
    %font=\itshape,
    topsep=.5ex,
    itemsep=0ex,
    leftmargin=1em,
    %itemindent=1em
}

\setlist[description]{
    labelindent=.4cm,
    style=unboxed,
    leftmargin=.4cm,
    font=\itshape,
    topsep=.5ex,
    itemsep=0ex
}

\lstset{columns=fullflexible,
        mathescape=true,
        literate=,
        numbersep=5pt,
        numberstyle=\tt\color{gray}
}
\lstset{escapechar=@,style=customc}

%space above and below
\lstset{aboveskip=2pt,belowskip=2pt}

\let\originalleft\left
\let\originalright\right
\renewcommand{\left}{\mathopen{}\mathclose\bgroup\originalleft}
\renewcommand{\right}{\aftergroup\egroup\originalright}

\expandafter\def\expandafter\normalsize\expandafter{%
    \normalsize
    \setlength\abovedisplayskip{2.5pt}
    \setlength\belowdisplayskip{2.5pt}
    \setlength\abovedisplayshortskip{2.5pt}
    \setlength\belowdisplayshortskip{2.5pt}
}
\title{Verification of Quantitative Network Properties}

\author{Kausik Subramanian, Loris D'Antoni, Aditya Akella}

\begin{document}

\maketitle

\begin{abstract}
In this work, we are exploring distributed verification and repair in 
legacy networks related to quantitative network policies.
\end{abstract}

\section{Introduction}
One of the key challenges of designing networks is verifying how good the network performs. This is further complicated by the fact that these networks are highly prone to failures and variable traffic demands. One of the common failure scenarios used to evaluate the network design and strategy is to quantify the relevant metrics like congestion under the occurrence  of k arbitrary link failures. While this is effective to understand the worst-case performance of the network, the probability of two links failing need not be uniform, and thus, we would want to estimate the expected value of the metric, given link failure probabilities. Similarly, the traffic characteristics modeled as a distribution can help us provide a realistic metric to evaluate the network, rather than using the tail characteristic for worst-case analysis. 

\section{Limitations of Existing Work}
Two works try to answer some of the questions pertaining to quantitative properties of the network. In this section, we will describe the approach and their limitations.
Probabilistic NetKAT~\cite{probnetkat} is an extension of NetKAT with support for probabilistic routing behaviors, for e.g., a switch can forward traffic to one of its neighbors based on a probability distribution. This can be used to express different routing strategies like ECMP, multi-path routing and oblivious TE. The language supports queries on these programs like expected congestion and latency. ProbNetKAT handles link failures by modeling the link as a program which probabilistically drops a packet (link failure probability). However, the ProbNetKAT language is not powerful enough to express more complex routing strategies which react to failures. The simplest example is a OSPF network, where traffic is forwarded along the shortest weighted path between the endpoints. When a link fails, the network recomputes the paths based on the functioning network topology. ProbNetKAT cannot be used to express the OSPF routing strategy. Traffic engineering strategies which adapt to failures to minimize an objective like MPLS tunneling also cannot be expressed using ProbNetKAT. Also, it is not clear if the approaches used in ProbNetKAT scale to larger networks and workloads.


Robust Validation of Network Designs: Chang et. al.~\cite{robustvalidation} propose a framework to validate network design under failures and uncertain demands. The framework formulates the validation problem as an optimization problem: to find the maximum value a metric can achieve under a set of link failure scenarios. An example metric is the Maximum Link Utilization. Importantly, this framework allows the specification of routing strategies which adapt based on the current network topology: Multi-commodity flow routing and MPLS-style tunneling. The limitation of this framework is that it performs a worst-case analysis of the network by finding the maximum value achieved by the metric. For example, if worst-case MLU of the network is > 1, it means that under a certain set of link failures, the network is oversubscribed. However, the occurrence of this particular scenario could be highly improbable, and the cost of repairing the over-subscription may not be reap significant benefits. Thus, an useful metric to validate the network design is to compute expected MLU based on the link failure probability distributions. Computing expected MLU in the framework would incur an exponential blowup due to the enumeration of failure scenarios.


Minesweeper~\cite{minesweeper} is a general approach to verification for “legacy” networks running distributed protocols like OSPF and BGP. It models the control plane interaction as logical constraints and uses SMT solvers to verify properties on the network like reachability, local equivalence etc. Minesweeper however is geared towards answering properties on a single traffic class, and do not tackle hyperproperties. Another framework is ARC~\cite{arc} which performs fast control plane analysis using a weighted graph representation, by which, it is able to encode network behavior under failures by ensuring that the path is the shortest in the graph. One of the limitations of ARC is that it cannot tackle all configuration constructs to create its weighted-graph representation. Both Minesweeper and ARC do not support verification of quantitative properties.


\section{Problem Statement}
Quantitative properties under failures for networks running legacy configurations:
Given a network N, set of network configurations C, traffic characteristics for a set of traffic classes T, and a set of failures F,  we want to answer questions of the form: 
\[
Q(N, C, T, F)
\]
Quantitative properties Q to consider:
\begin{itemize} 
\item Congestion-freeness: Under any arbitrary k link failures, is the network congestion-free, i.e., flows do not exceed any link’s maximum capacity.
\item Latency-Inflation: Under any arbitrary k failures, is latency bounded for every flow? We can model latency by hop-counts. Making latency load-aware would be interesting, but difficult to model and verify.
\item Rate-limiting: Verify if in-network rate-limiting is performed correctly even under failures.
\end{itemize}
We assume here that failures have uniform probability distributions (second question uses the actual probabilities). Thus, we need to explore potentially exponential failure scenarios, we plan to investigate distributed verification by first clustering the failure events and exploring these clusters in parallel. 

\subsection{Identifying Most Likely Most Impacting Failure Scenarios}
Given a set of policies, let the weight function W be the penalty incurred by the topology N due to policy violations. We want to find the top-L failure scenarios with probability greater than threshold  with respect to the penalty incurred under each topology with the failed links (< k link failures). Here, we can use the link failure probabilities to find critical link failure events which cause the most violation of intents. Ideally, our approach should be general enough for independent probability distributions (i.e., failure of a link does not affect another link’s failure probability) and correlated failure probability distributions (i.e., link failures depend on each other. A node failure can be treated as a special case of correlated link failure where all links connected to a node will fail). 

\subsection{Short-term and Long-term Repair}
While quantitative verification is important to understand the network behavior under failures, it does not provide solutions for the network operator. Thus, it is crucial to use the verification insights to suggest changes to the network configuration. We intend to investigate two types of network repair: One geared towards short-term traffic characteristics (can use spikes in traffic) by changing routing configurations to improve network resilience.  Using long-term traffic characteristics, we would propose changes to network topology (adding new links, upgrading link capacities) to ensure the network behaves smoothly under a bounded number of failures.

\section{Preliminary Approach}
We want to leverage the weighted-graph abstract representation of the control plane introduced in ARC as a starting point. Building on ARC, we want to formulate ILP constraints to verify quantitative properties dealing with multiple traffic classes. This will be augmented with real-world failure probability and traffic characteristics to perform more accurate verification. We formulate an initial model of the problem next.

\noindent\paragraph{Input}
\begin{itemize}
\item Network Topology $N = (V, L)$: set of routers and links. 
\item ETGs constructed from network configurations $C$ (OSPF/BGP/Static/ACLs/...)
\item Set of flows where 
each flow $f \in FLOWS$ is of the form: $<$flow, srcRouter, dstRouter, traffic$>$\
\item Link Probability Failures $P_f(r_1, r_2)$
\end{itemize}
\noindent\paragraph{Variables}
\begin{itemize}
\item Link Failure Variable $F(r_1, r_2)$ which is set to 1 if link $r_1 - r_2$ fails.

\item Best Route $R_{best}(r, f)$: This variable is used to store the shortest route  
of all input routes received at router $r$ for flow $f$.

\item Network Forwarding $Fwd$: The control plane will decide the next-hop 
router based on the best route received at a router.  
$Fwd(r_1, r_2, f)== 1$ implies that router $r_1$ will forward flow $f$ to router $r_2$.

\item Link Congestion $\omega(r_1, r_2)$: Binary variable indicating if a 
link is congested or not.  
%We can model Congestion of link $r_1 \rightarrow r_2$ as $C(r_1, r_2) = \frac{}{\sum Fwd(r_1, r_2, p) * Traffic(p)$ 
\end{itemize}

\noindent\paragraph{Constraints}
We present the integer linear constraints to verify if the network 
is congestion-free under $k$ arbitrary link failures. This is 
enforced by the following constraint:
\begin{equation}
	\forall r_1, r_2.~(r_1,r_2) \in L. ~\sum_{(r_1, r_2)} F(r_1, r_2) \leq k
\end{equation}

For a flow $f$, at any vertex $v$ of the ETG, the best route to the
destination will the shortest active path through of its neighbors ($N(v,f)$).
We can inductively express this as: 
\begin{equation}
	\forall n \in N(v,f).~ R_{best}(v, f) \leq R_{best} (n, f) + W(v, n, f) + \infty \times F(v,n) 
\end{equation}
where $W(v,n,f)$ denotes the ETG edge weight and $\infty$ is a large constant. These
constraints ensure $R_{best}(v, f)$ is the smallest of all the active routes at $v$.

The constructed ETG have the following property: forwarding happens along the
active shortest path. Thus, forwarding will happen along the link which sends
the best advertisement (shortest in this case). This is expressed as follows:
\begin{equation}
	\forall n \in N(v,f). ~R_{best}(v,f) + \infty \times (1 - Fwd(v,n,f)) \geq R_{best} (n, f) + W(v, n, f)
\end{equation}
These constraints ensure that $Fwd(v,n,f)$ is 1 along the best path.

Since forwarding cannot happen along failed links, we add the following 
constraints to ensure correct forwarding semantics:
\begin{equation}
\forall n \in N(v,f).~Fwd(v,n,f) + F(v,n) \leq 1 
\end{equation}
Thus, if link $v-n$ is failed, flow $f$ will not be forwarded through the link.

\noindent\paragraph{Output} 
To check for congestion, we first add the following constraints for 
each link to indicate congestion when utilization exceeds capacity: 
\begin{equation}
\forall (r_1, r_2) \in L. ~~\frac{\sum_{f \in F} Fwd(r_1, r_2, f) * TC(f)}{Cap(r_1, r_2)} - \omega(r_1, r_2) \geq 0
\end{equation}
Thus, if traffic on link does not exceed capacity of the link, the congestion value
omega will be 0. Also, if $\omega$ is 1, then the link has to be congested. Therefore,
to check for a particular link congestion, we can have the objective as:
\[
\texttt{maximize} ~~\omega(r_1, r_2)
\]	
To find if any link in the network can become congested, we can add the following
objective:
\[
\texttt{maximize} ~~\sum_{\forall (r_1,r_2) \in L} \omega(r_1, r_2)
\]


\textbf{Most Likely Critical Event}: We modify our question of congestion from above to
incorporate failure probabilities for link failure scenarios $(r_1, r_1'),\ldots(r_{k-1}, r_{k-1}')$ : 
\[
\texttt{maximize} (P_f(r_1,r_1') *\ldots P_f(r_{k-1}, r_{k-1}')). 
~~\exists r_1, r_2. \omega(r_1, r_2) = 1
\]
Product can be expressed as a sum of logarithms of probabilities to ensure the objective is linear. 

%%!TEX root = paper.tex
\begin{theorem}
Given a topology $T = (R, L)$, the ARC, the set of  
traffic characteristics $\{tc \in TC | tc \in IP \times R \times R \times \nat\}$ and 
$(L_f, \fail)$ where $L_f \subseteq L$ denotes the subset of links 
which can potentially fail and $\fail$ denotes the number of links which have failed,
the problem of finding a set of links $\links \subseteq L_f$ such that $|\links| = \fail$
and there exists a link $l \in L \setminus \links$ where the sum of flows exceeds capacity of the link $l$
\end{theorem}

\begin{proof}
We show that the Hamiltonian path problem, which is NP-complete reduces to the
quantitative verification problem under failures. The latter is also in NP, so after the reduction we 
 can conclude that it is also NP-complete.
 
\end{proof} 


\bibliographystyle{abbrv} 
\begin{small}
	\bibliography{refs}
\end{small}

\end{document}