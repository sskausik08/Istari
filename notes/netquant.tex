\documentclass[]{article}
\usepackage{graphicx}
\usepackage{tikz}
\usepackage{xspace}
\usepackage{enumitem}
\usepackage{mathtools}
\usepackage{cleveref}
\usepackage{listings}
\usepackage{amsmath}
\usepackage{amssymb}
\usetikzlibrary{calc,automata,positioning}
\def\istari{\textsc{Istari}\xspace}
\usepackage{float}
%opening
\newcommand{\minisection}[1]{\noindent{\bf #1.}}
\newcommand{\secref}[1]{{\S\ref{#1}}}
\newcommand{\secsref}[2]{{Sections~\ref{#1} and \S\ref{#2}}}
\newcommand{\figref}[1]{{Figure~\ref{#1}}}
\newcommand{\figsref}[2]{{Figure~\ref{#1} and \ref{#2}}}
\newcommand{\appref}[1]{{Appendix~\ref{#1}}}
\newcommand{\thmref}[1]{{Theorem~\ref{#1}}}
\newcommand{\lemref}[1]{{Lemma~\ref{#1}}}
\newcommand{\corref}[1]{{Corollary~\ref{#1}}}
\newcommand{\proref}[1]{{Property~\ref{#1}}}
\newcommand{\tabref}[1]{{Table~\ref{#1}}}

\newcommand{\compactcaption}[1]{\vspace{-0.5em}\caption{#1}\vspace{-0.5em}}
\newcommand{\botcompactcaption}[1]{\caption{#1}\vspace{-1em}}
\newcommand{\topcompactcaption}[1]{\vspace{-1em}\caption{#1}\vspace{-0.5em}}

%\newtheorem{definition}{Definition}

\newenvironment{compactitemize}
{
   \begin{itemize}[leftmargin=1.5em]
   \vspace{-1ex}
   \setlength{\topsep}{0pt}
   \setlength{\itemsep}{0em}
   \setlength{\parskip}{0pt}
   \setlength{\parsep}{0pt}
}
{
   \vspace{-1ex}
   \end{itemize}
}
\newenvironment{compact2itemize}
{
	\begin{itemize}[leftmargin=1.5em]
		\vspace{-1ex}
		\setlength{\topsep}{0pt}
		\setlength{\itemsep}{0.5em}
		\setlength{\parskip}{0pt}
		\setlength{\parsep}{0pt}
	}
	{
		\vspace{-1ex}
	\end{itemize}
}


\newenvironment{compactenumerate}
{
   \begin{enumerate}[leftmargin=1.5em]
   \vspace{-1ex}
   \setlength{\topsep}{0pt}
   \setlength{\itemsep}{0em}
   \setlength{\parskip}{0pt}
   \setlength{\parsep}{0pt}
}
{
   \vspace{-1ex}
   \end{enumerate}
}

\def\rat{\mathbb{Q}}
\def\real{\mathbb{R}}
\def\nat{\mathbb{N}}
\def\route{\mathfrak{R}}
\def\paths{\mathfrak{P}}
\def\links{\mathbb{L}}
\def\waypt{\mathbb{W}}


%-----
%Space tricks

%\algtext*{EndWhile}% Remove "end while" text
%\algtext*{EndIf}% Remove "end if" text
%\algtext*{EndProcedure}% Remove "end while" text
%\algtext*{EndFor}% Remove "end while" text

\makeatletter
\lst@InstallKeywords k{attributes}{attributestyle}\slshape{attributestyle}{}ld
\makeatother
\newcommand\gtt[1]{{\color{gray}\texttt{\textbf{#1}}}}
\newcommand\nf[1]{{\normalfont#1}}

\lstdefinestyle{customc}{
  belowcaptionskip=1\baselineskip,
  breaklines=true,
  xleftmargin=\parindent,
  language=C,
  showstringspaces=false,
  basicstyle=\small\ttfamily,
  keywordstyle=\small\bfseries\ttfamily\color{NavyBlue},
  commentstyle=\itshape\color{black},
  identifierstyle=\ttfamily\color{black},
  stringstyle=\itshape\color{NavyBlue},
  keywords={ map, flatMap, reduce, return, define, gaussian, gauss, step, elif, then, skip, if, else, reduceByKey, sensitiveAttribute, fairnessTarget, def},
moreattributes={ let, where}, % etc...
attributestyle = \itshape\ttfamily\color{attributecolor}
}

\setlist[itemize]{
    %labelindent=.4cm,
    %style=unboxed,
    %leftmargin=.4cm,
    %font=\itshape,
    topsep=.5ex,
    itemsep=0ex,
    leftmargin=1em,
    %itemindent=1em
}

\setlist[description]{
    labelindent=.4cm,
    style=unboxed,
    leftmargin=.4cm,
    font=\itshape,
    topsep=.5ex,
    itemsep=0ex
}

\lstset{columns=fullflexible,
        mathescape=true,
        literate=,
        numbersep=5pt,
        numberstyle=\tt\color{gray}
}
\lstset{escapechar=@,style=customc}

%space above and below
\lstset{aboveskip=2pt,belowskip=2pt}

\let\originalleft\left
\let\originalright\right
\renewcommand{\left}{\mathopen{}\mathclose\bgroup\originalleft}
\renewcommand{\right}{\aftergroup\egroup\originalright}

\expandafter\def\expandafter\normalsize\expandafter{%
    \normalsize
    \setlength\abovedisplayskip{2.5pt}
    \setlength\belowdisplayskip{2.5pt}
    \setlength\abovedisplayshortskip{2.5pt}
    \setlength\belowdisplayshortskip{2.5pt}
}
\title{Distributed Verification of Quantitative Network Properties}

\author{Kausik Subramanian, Loris D'Antoni, Aditya Akella}

\begin{document}

\maketitle

\begin{abstract}
We investigate distributed verification and synthesis in 
networks related to quantitative network policies.
\end{abstract}

\section{Introduction}
One of the key challenges of designing networks is verifying
how good the network performs. This is further complicated 
by the fact that these networks are highly prone to failures
and variable traffic demands. One of the common failure scenarios
used to evaluate the network design and strategy is to quantify 
the relevant metrics like congestion under the occurence  
of $t$ arbitrary link failures. While this is effective to understand
the worst-case performance of the network, the probability
of two links failing need not be uniform, and thus, we would 
want to estimate the expected value of the metric, given
link failure probabilities. Similarily, the traffic characteristics
modelled as a distribution can help us provide a realistic
metric to evaluate the network, rather than using the tail
characteristic for worst-case analysis.

\section{Limitations of Existing Work}
Two works try to answer some of the questions pertaining to
quantitative properties of the network. In this section, we 
will describe the approach and their limitations.

\paragraph{Probabilistic NetKAT}
Probabilistic NetKAT~\cite{probnetkat, probnetkatpopl17} 
is an extension of NetKAT with support
for probabilistic routing behaviors, for e.g., a switch 
can forward traffic to one of its neighbors based on a 
probability distribution. This can be used to express different
routing strategies like ECMP, multi-path routing and 
oblivious TE. The language supports queries 
on these programs like expected congestion and latency. ProbNetKAT
handles link failures by modelling the link as a program
which probabilistically drops a packet (link failure
probability).

However, the ProbNetKAT language is not powerful enough to 
express more complex routing strategies which react to failures. 
The simplest example is a OSPF network, where traffic is 
forwarded along the shortest weighted path between the endpoints. 
When a link fails, the network recomputes the paths based on the 
functioning network topology. ProbNetKAT cannot be used to express
the OSPF routing strategy. Traffic engineering strategies which adapt 
to failures to minimize an objective like MPLS tunneling also 
cannot be expressed using ProbNetKAT. Also, it is not clear 
if the approaches used in ProbNetKAT scale to larger networks 
and workloads.

\paragraph{Robust Validation of Network Designs}
Chang et. al~\cite{robustvalidation} propose a framework 
to validate network design under failures and uncertain demands. 
The framework formulates the validation problem as an 
optimization problem: to find the maximum value a metric can
achieve under a set of link failure scenarios. An example 
metric is the Maximum Link Utilisation. Importantly,
this framework allows the specification of routing strategies
which adapt based on the current network topology: Multi-commodity
flow routing and MPLS-stlye tunneling. 

The limitation of this framework is that it performs a 
worst-case analysis of the network by finding the maximum value 
achieved by the metric. For example, if worst-case 
MLU of the network is $> 1$,
it means that under a certain set of link failures, the network is
oversubscribed. However, the occurence of this particular scenario
could be highly improbable, and the cost of repairing the 
oversubscription may not be reap significant benefits. 
Thus, an useful metric
to validate the network design is to 
compute expected MLU based on the 
link failure probability distributions. 
Computing expected 
MLU in the framework 
would incur an expotential blowup due to the
enumeration of failure scenarios.

\section{Problem Statement}
\paragraph{Input.} We are provided as input a 
network topology $T=(V,L)$, an adaptive 
routing strategy $R(T_f)$
parameterized over the current functioning topology $T_f$
which decides the paths taken by the traffic classes
in $T_f$. Let us define the 
set of traffic classes $\Lambda 
\subseteq V \times V$. For each $\lambda \in \Lambda$,
we are given a traffic probability 
distribution function 
$\omega(\lambda)$
(discrete or continous?). To quantify behavior under
failures, we have a link failure probability function
$\mu: L \rightarrow [0,1]$. Let us define $X$ to be 
the set of failure scenarios (bounded by $k$ links).

\paragraph{Output.} Let us define the
metric $M$ which takes an input a topology and a routing 
strategy; it returns a 
real number (for example, max link utilization). 
The question is as follows: for a given $T$ and $R$,
can we compute: 
\[
	E[M] = \sum_{x \in X} \mu(x). M(T_x, R(T_x))
\]

\paragraph{Verification Approach.} The naive brute-force 
approach is to enumerate all link failures $x \in X$
and compute $M(T_x, R(T_x)$, which can then be aggregated to 
compute $E[M]$. This process is highly parallelisable (each 
$x$ can be potentially run on a different machine). Thus, we
propose a distributed verification framework to speed and 
scale up the queries. The other aspect to investigate is 
how to partition the set of failures $x \in X$ across
 the machines such that each partition can be computed
 efficiently. This could involve approximations with 
 provable guarantees on the distance from the actual 
 value computed for the partition. The other aspect 
 is to leverage the link failure probability function
 $\mu$ to avoid enumerating all possible link failures. 

\paragraph{Synthesis.} While it is useful for operators
to compute metrics to validate network design, operators need
to repair the network (by changing routing strategies, or 
provisioning new links) to ensure the metric does not violate
a certain threshold (for e.g. a cloud provider can provide 
an SLO of 99.99\% availability). Thus, we investigate 
if we can suggest repairs along with the verification computation
in a distributed manner. 
 
\bibliographystyle{abbrv} 
\begin{small}
	\bibliography{refs}
\end{small}

\end{document}