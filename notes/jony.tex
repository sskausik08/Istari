\documentclass[]{article}
\usepackage{graphicx}
\usepackage{tikz}
\usepackage{xspace}
\usepackage{enumitem}
\usepackage{mathtools}
\usepackage{cleveref}
\usepackage{listings}
\usepackage{amsmath}
\usetikzlibrary{calc,automata,positioning}
\def\genesis{\textsc{Genesis}\xspace}
\def\zeppelin{\textsc{Zeppelin}\xspace}
\def\qs{\textsc{Quicksilver}\xspace}
\usepackage{float}
%opening
\title{Queries on Encrypted Data on Public Clouds}

\author{Kausik Subramanian}

\begin{document}

\maketitle

\begin{abstract}
We investigate running queries on encrypted data on public clouds using partial decryption functions. 
\end{abstract}

\section{Introduction}
Given the encryption and decryption function, and the query (or Spark program) 
on the client-side, we find an equivalent program which can be run on the encrypted
data on the untrusted server, such that we do not leak any data from the 
query or its execution on the server. 

\section{Motivating Example}
Consider the following SQL query to run on the encrypted table \texttt{Customers}:
\begin{verbatim}
SELECT SUM(sales)
FROM Customers; 
\end{verbatim}
Systems like CryptDB~\cite{cryptdb}, Monomi~\cite{monomi} 
and Seabed~\cite{seabed} will use a additive homomorphic
encryption function for the column data \texttt{sales} to
enable aggregation over the encrypted data and compute the
encrypted \texttt{SUM(sales)}. 

\noindent Consider a predicated aggregation query:
\begin{verbatim}
SELECT SUM(sales)
FROM Customers
WHERE sales < 1000; 
\end{verbatim}
The \texttt{sales} column is encrypted using the ASHE encryption 
scheme~\cite{seabed} which is not order-preserving, i.e.,
$m_1 > m_2 \not\Rightarrow E(m_1) > E(m_2)$, thus, we cannot
perform the range query over the encrypted \texttt{sales} data. 

The user, thus has two choices: (1) The user can supply the 
decryption function with the secret key as part of the query
so that the server can perform the predicated aggregation query,
or (2) the user downloads the entire \texttt{sales} data and 
perform the predicated aggregation locally. 
\end{document}